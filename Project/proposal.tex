\documentclass{article}
\usepackage{geometry}
\usepackage{hyperref}
\geometry{a4paper, margin=1in}

\title{Course Project Proposal}
\author{Team Name}
\date{\today}

\begin{document}

\maketitle

\section*{1. Project Title}
\textbf{Predictive Maintenance Using Machine Learning for Industrial Equipment}

\section*{2. Team Members}
\begin{itemize}
    \item Name 1 (Student ID, Email)

\end{itemize}

\section*{3. Objective}
The objective of this project is to build a predictive model using machine learning techniques that can forecast machine failure in industrial settings, reducing downtime and maintenance costs.

\section*{4. Background and Motivation}
Industrial equipment downtime can lead to significant financial losses. Predictive maintenance leverages AI to address this challenge by predicting failures before they occur, improving operational efficiency. Machine learning models trained on historical data can detect patterns associated with failures, helping organizations avoid costly downtime.

\section*{5. Proposed Methodology}
We will approach the project with the following steps:

\subsection*{Data Collection}
We will collect equipment sensor data from a publicly available industrial dataset, such as those found on the UCI Machine Learning Repository or Kaggle.

\subsection*{Algorithms/Models}
We plan to implement decision trees, random forests, and neural networks to predict equipment failures. We will experiment with different algorithms and select the one that provides the best performance.

\subsection*{Tools and Technologies}
We will use Python with libraries such as \texttt{scikit-learn}, \texttt{TensorFlow}, and \texttt{Pandas} for data preprocessing, model building, and evaluation.

\subsection*{Evaluation Metrics}
The success of the models will be evaluated using metrics such as accuracy, precision, recall, F1-score, and ROC-AUC curve, with a focus on balancing recall and precision for imbalanced data.

\section*{7. Expected Outcomes}
We expect to develop a predictive model with an accuracy of at least 90\% in identifying equipment failures. One potential challenge is handling imbalanced data, as machine failures are rare events compared to normal operation.

\section*{8. References}
\begin{itemize}
    \item Predictive Maintenance using Machine Learning, Journal XYZ, 2022.
    \item UCI Machine Learning Repository: \url{https://archive.ics.uci.edu/ml/index.php}
\end{itemize}

\section*{9. Conclusion}
By the end of this course project, we aim to have a fully functional predictive maintenance model that can forecast equipment failures in real-time, potentially contributing to efficiency improvements in industrial maintenance strategies.

\end{document}
